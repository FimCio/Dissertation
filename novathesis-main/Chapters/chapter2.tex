%!TEX root = ../template.tex
%%%%%%%%%%%%%%%%%%%%%%%%%%%%%%%%%%%%%%%%%%%%%%%%%%%%%%%%%%%%%%%%%%%%
%% chapter2.tex
%% NOVA thesis document file
%%
%% Chapter with the template manual
%%%%%%%%%%%%%%%%%%%%%%%%%%%%%%%%%%%%%%%%%%%%%%%%%%%%%%%%%%%%%%%%%%%%

\typeout{NT FILE chapter2.tex}%

\chapter{State Of the Art}
\label{cha:users_manual}


\section{Introduction}
\label{sec:introduction}

The objective of this chapter is to establish a solid foundation for the research by reviewing the existing literature on software bug triage. This review covers the evolution of bug triage methods, the challenges of traditional approaches, and the impact of using machine learning and artificial intelligence to improve these processes.

The chapter highlights the limitations of manual bug triage, such as long response times and frequent miss-classifications, which make it inefficient. It then explores how AI and machine learning have been used to automate and enhance bug classification, prioritisation, and assignment, ultimately improving accuracy and reducing manual effort.

\section{Overview of Bug Triage}
\label{sec:overview_of_bug_triage}

Bug triage is a critical process in software development aimed at managing and prioritising bug reports to ensure efficient allocation of resources for their resolution. With the increasing complexity of modern software systems and the growing volume of bug reports generated during development and maintenance, effective bug triage has become essential for maintaining software quality and ensuring timely delivery of updates.

Traditionally, bug triage has been a manual process, where a human triager evaluates each bug report to determine its severity, priority, and the development team or individual responsible for addressing it. While this approach relies heavily on the expertise of the triager, it is prone to several challenges, including inefficiencies due to the time required for manual evaluation, inconsistencies arising from subjective decisions, and delays in resolution when reports are misclassified or improperly assigned.

The emergence of automated approaches to bug triage, supported by advancements in artificial intelligence (AI) and machine learning (ML), has provided opportunities to address these challenges. Automated bug triage leverages computational models to classify and assign bug reports based on features such as textual descriptions, metadata, and historical bug data. These methods aim to improve the accuracy and speed of bug report processing while reducing the reliance on human intervention.

This section provides a foundational understanding of the bug triage process and highlights its importance in the software development lifecycle. The subsequent sections explore the application of AI/ML techniques, data preprocessing strategies, and model selection approaches to enhance the bug triage process. By understanding the challenges and opportunities within bug triage, this research aims to contribute to the development of more efficient and scalable solutions.



\section{Literature review}
\label{sec:literature_review}

\subsection{Bug Prioritization}


In every bug description should be mentioned the bug priority. Bug priority represents the urgency of the bug, according to which the bug will be attended. Often ranges from high to low. High are critical issues that needs immediate attention and low are issues that can be resolved later and have meaningful impact. Prioritization is also a problem in SBT, since it is a manual, time and resource consuming task that consists in assign a priority to a bug \cite{Uddin2017}. 

Umer et al. \cite{Umer2020} proposed a solution using CNN-based automatic approach to multi-class priority prediction of bug reports. The approach uses deep learning model, natural language techniques and emotion analysis for bug report's priority prediction.

\subsection{Other Approaches}

		A comprehensive review of Artificial Intelligence (AI) techniques used for Software Bug Triaging (SBT) is presented in the paper "An Artificial Intelligence Framework on Software Bug Triaging" \cite{Nagwani2023} . The authors analysed which AI-based method is more efficient in SBT. In the paper are evaluated 6 AI-based techniques that were mostly used in 123 studies: machine learning, information retrieval, social network analysis, recommender systems, mathematical modelling and optimisation, and deep learning. The findings of the paper indicate that:
	\begin{itemize}
		\item The Machine Learning and Information Retrieval techniques are widely applied in early SBT methods and still are important for classification and bug report's text analysis.
		\item The Deep Learning Approaches especially those using word embeddings like Word2Vec and Glove offer a better accuracy. 
		\item Top-10 accuracy of 86\% was achieved using Deep Learning Models with Word2Vec and Glove embeddings across various repositories (e.g., Eclipse, Mozilla, and Gento). 
	\end{itemize}
	
		The effectiveness from combination of various word embeddings and deep learning models for the SBT was studied in the paper "An empirical Assessment of Different Word Embedding and Deep Learning Models for Bug Assignment" \cite{Huang2024}, the authors tested combinations of word embeddings with Deep Learning (DL) classifiers. Word embeddings selected: Word2Vec, Glove, Next-bug, ELMo, and BERT. For deep learning classifiers were selected: Text-CNN, LSTM, Bi-LSTM, LSTM with attention, Bi-LSTM with attention, MLP, and Native Bayes. As a result of the study from the 35 possible combinations the most successful combination was Bi-LSTM with attention combined with ELMo embeddings that had the best performance when referring to top-k accuracy measurements. For that combination was possible to achieve 46\% for Eclipse JDT and 28\% on Mozilla for Top-1 accuracy. For top-5 accuracy was obtained 78\% on Eclipse and 68\% on Mozilla. 
 
	A novel technique of SBT using deep reinforcement learning (BT-RL) was suggested in the paper "Automatic Bug Triaging via Deep Reinforcement Learning" \cite{Liu2022}.
	In the technique is used:
	\begin{itemize}
		\item Deep Multi-Semantic Feature (DMSF)fusion model that uses Recurrent Neural Network (RNN) to extract only the relevant information from the bug report.
		\item On-line Dynamic Matching (ODM) model, that by using reinforcement learning, this component learns form historical data of bug fixes and form developer activities. It matches the new bug reports with developers based on a probabilistic model, allowing to assign developers in a more dynamic way as it learns.
	\end{itemize}
	The paper used data sets form: NetBeans, Mozilla, Eclipse and OpenOffice. The highest Top-k accuracy that was achieved was for the Eclipse data set with a value of 78\% for Top-5 and 46\% for Top-1.

	
	In the paper "Applying Convolutional Neural Networks With Different Word Representation Techniques to Recommend Bug Fixers" () the authors propose a deep learning-based bug triage technique using a convolutional neural network (CNN) with three different word representation techniques: Word to Vector (Word2Vec ), Global Vector (GloVe), and Embeddings from Language Models (ELMo). 
	
	 



