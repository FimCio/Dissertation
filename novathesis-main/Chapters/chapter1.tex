%!TEX root = ../template.tex
%%%%%%%%%%%%%%%%%%%%%%%%%%%%%%%%%%%%%%%%%%%%%%%%%%%%%%%%%%%%%%%%%%%
%% chapter1.tex
%% NOVA thesis document file
%%
%% Chapter with introduction
%%%%%%%%%%%%%%%%%%%%%%%%%%%%%%%%%%%%%%%%%%%%%%%%%%%%%%%%%%%%%%%%%%%

\typeout{NT FILE chapter1.tex}%

\chapter{Introduction}
\label{cha:introduction}


\section{Welcome to the \novathesis\ Template}
\label{sec:if_you_use_this_template}

In software development, developers and software engineers are not only responsible for implementing new software requirements but must also address bugs. Bugs are errors in the software code that make it unreliable, often leading to serious issues. A notable example occurred in 1996 when the European Space Agency’s Ariane 5 rocket, valued at nearly \$370 million, was destroyed 36 seconds after liftoff due to a software bug. According to the Consortium for IT Software Quality (CISQ), software bugs have cost the United States \$2.41 trillion up until 2022. In Europe, companies lose billions annually due to these errors. 

Bug analysis is among the most time-intensive activities in software development, with developers spending 60-70\% of their time on code maintenance[reference]. Another similarly time-consuming task is software bug triage (SBT), which involves reviewing bug reports and assigning them to the correct team or developer. SBT can be particularly challenging when bug reports are poorly described or when team responsibilities overlap. This can lead to bugs being passed back and forth between teams, delaying resolution and driving up project costs.

A potential way to cut down the time spent on software bug triage (SBT) is by leveraging Artificial Intelligence (AI) and Machine Learning (ML). Over the past decade, AI and ML have shown great success in automating and optimizing various tasks, including analysis. By implementing a machine learning algorithm that extracts key terms from bug reports, it's possible to identify which team or developer should handle the issue based on those keywords. This approach could streamline the triage process and significantly reduce the time involved.



