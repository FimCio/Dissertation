%!TEX root = ../template.tex
%%%%%%%%%%%%%%%%%%%%%%%%%%%%%%%%%%%%%%%%%%%%%%%%%%%%%%%%%%%%%%%%%%%
%% chapter1.tex
%% NOVA thesis document file
%%
%% Chapter with introduction
%%%%%%%%%%%%%%%%%%%%%%%%%%%%%%%%%%%%%%%%%%%%%%%%%%%%%%%%%%%%%%%%%%%

\typeout{NT FILE chapter1.tex}%

\chapter{Introduction}
\label{cha:introduction}


\section{Introduction}
\label{sec:if_you_use_this_template}

In software development, developers and software engineers are not only responsible for implementing new software requirements but must also address bugs. Bugs are errors in the software code that make it unreliable, often leading to serious issues. Software bugs costs companies over 45\% of the budget \cite{Xuan2015}. Nearly 70\% of the developers time is spent debugging \cite{Minelli2015}.A large contribution to that value are the wrongly assigned bugs.

 A notable example occurred in 1996 when the European Space Agency's Ariane 5 rocket, valued at nearly \$370 million, was destroyed 36 seconds after lift-off due to a software bug. According to the Consortium for IT Software Quality (CISQ), software bugs have cost the United States \$2.41 trillion up until 2022. In Europe, companies lose billions annually due to these errors. 

A potential way to cut down the time spent on software bug triage (SBT) is by leveraging Artificial Intelligence (AI) and Machine Learning (ML). Over the past decade, AI and ML have shown great success in automating and optimizing various tasks, including analysis. By implementing a machine learning algorithm that extracts key terms from bug reports, it's possible to identify which team or developer should handle the issue based on those keywords. This approach could streamline the triage process and significantly reduce the time involved.



