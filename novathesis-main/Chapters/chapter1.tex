%!TEX root = ../template.tex
%%%%%%%%%%%%%%%%%%%%%%%%%%%%%%%%%%%%%%%%%%%%%%%%%%%%%%%%%%%%%%%%%%%
%% chapter1.tex
%% NOVA thesis document file
%%
%% Chapter with introduction
%%%%%%%%%%%%%%%%%%%%%%%%%%%%%%%%%%%%%%%%%%%%%%%%%%%%%%%%%%%%%%%%%%%

\typeout{NT FILE chapter1.tex}%

\chapter{Introduction}
\label{cha:introduction}


\section{Motivation}
\label{sec:motivation}

In software development, developers and software engineers are not only responsible for implementing new software requirements but must also address bugs. Bugs are errors in the software code that make it unreliable, often leading to serious issues.  Nearly 70\% of the developers time and of the company budget is spent debugging \cite{Minelli2015, Xuan2015, Boehm2011, Hooimeijer2007}.

Delays and unsolved software bugs can lead to serious and very expensive issues. In 1996, for instance, the European Space Agency's Ariane 5 rocket, which was worth around \$370 million, was destroyed 36 seconds after lift-off because of a software error \cite{Ariane5}. The Consortium for IT Software Quality (CISQ) estimates that only in 2020, software problems cost the US over \$2.08 trillion. 

One crucial step in the software development process is Software Bug Triage (SBT). The evaluation of each problem report's severity, priority, and relevance to the developer's competence has historically been done manually, which takes a lot of time and experience. Manual triaging has grown to be a bottleneck due to the complexity and volume of software projects, which can cause delays in problem fixes and even lower software quality. 

By taking factors like developer experience, workload, and past performance, the incorporation of machine learning and artificial intelligence (ML) into bug triage may not only speeds up the assignment process but also improves accuracy. This technical progress is essential for handling the increasing demands of contemporary software development, guaranteeing prompt and efficient bug fixes, and upholding strict software quality standards.


\section{Objectives}
\label{sec:objectives}

Trabalho pretendido ... 

- extracao dos dados 
- extrair valor dos dados (interpretar o texto) (completar na introducao)
- estudar varios modelos 
- 
- implementar um sistema 


\section{Document Structure}
\label{sec:doc_structure}

- explicar o conteudo de cada capitulo 